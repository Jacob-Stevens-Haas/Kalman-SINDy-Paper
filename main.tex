\documentclass{article}[letterpaper]
\usepackage[utf8]{inputenc}
\usepackage{amsmath}
\usepackage{amssymb}
\usepackage{xcolor}
\usepackage{natbib}

\newcommand{\red}[1]{\textcolor{red}{#1}}

\title{Learning Nonlinear Dynamics Using Kalman Smoothing}
\author{jmsh}
\date{October 2023}

\begin{document}

\maketitle

\abstract{Identifying ODEs from measurement data requires both fitting the dynamics and assimilating the measurement data.  The Sparse Identification of Nonlinear Dynamics (SINDy) method thus involves a derivative estimation (and optionally, smoothing) step and a sparse regression on ODE terms.  Kalman smoothing provides a useful framework for assimilating the measurement data whose noise behavior is well understood.  Previously, derivatives in SINDy and its python package, pysindy, had been estimated by finite difference, L1 total variation minimization, or local filters like Savitsky-Golay.  However, Kalman discovers ODEs that best recreate the essential dynamics in simulation, and when combined with hyperparameter optimization, requires the least amount of tuning.  The authors have incorporated Kalman smoothing into the existing pysindy architecture, allowing for rapid adoption of the method.}

\section{Introduction}
The method of Sparse Identification of Nonlinear Dynamics, or SINDy, seeks to discover a differential equation governing an arbitrary, measured system.  The method takes as input some coordinate measurements over time, such as angles between molecular bonds \red{(Boninsegna article)} or a spatial field, such as wave heights \red{(Rudy article)}, and returns the best ordinary or partial differential equation (ODE or PDE) from a library.  However, the method struggles to accomodate significant measurement noise.  On the other hand, Kalman theory has a half-century history of assimilating noisy data to smooth a trajectory.  Its noise properties have been well studied. The paper integrates mature, well-established Kalman theory with emerging SINDy technology and generalized cross validation parameter selection.  It finds that the combination is competitive with other combinations of data smoothing and system identification, and has an advantage in preservation of problem structure and ease of parameter selection.

In section two, we describe the individual methods of SINDy and Kalman smoothing, providing some literature review.  In section three, we demonstrate the advantages of incorporating Kalman with SINDy in a series of experiments.  We conclude with avenues for future research in section four.

\begin{itemize}
    \item Background on SINDy (relevant variants: WeakSINDy, Ensemble Kalman Identification)
    \item TOC paragraph
\end{itemize}
*Smoothing and differentiation
\section{Background}
\begin{itemize}
    \item Kalman smoother description.  Figure: graphics of kalman smoothing
    \item SINDy description.  Figure: Graphics of SINDy
    \item Existing Kalman System Identification.
\end{itemize}
\section{Experiments}
Methods:
\begin{itemize}
    \item table
    \item Compare with (a) WeakSINDy, (b) TV, SG filter
    \item metrics: F1, MAE of coefficients (\textcolor{red}{or MSE?})
    \item Utilize MIOSR optimizer (\textcolor{red}{with Ensembling wrapper?})
\end{itemize}
ODE data sets \textcolor{red}{Do I add {\it all} the possible ones I have set up (Rossler, Duffing)?  Do I add the reaction network one?}
\begin{itemize}
    \item Hudson Bay Company Lotka-Volterra
    \item Van der Pol oscillator
    \item Hopf
    \item MHD \textcolor{red}{Would need to add}
\end{itemize}
PDE Data sets \textcolor{red}{anything real?}
\begin{itemize}
    \item Inviscid Burgers
    \item KdV
    \item nonlinear Schrodinger
    \item KS
    \item Reaction-Diffusion
\end{itemize}

\subsection{Noise tolerance and data length}
\textcolor{blue}{Figure: plot of score for each method across range of smoothing parameters.  subplots for each ODE \& metric}.  \textcolor{red}{Parameter-search wrapper-experiment inside P-search comparison experiment}.
% \subsection{ODEs: Robust Kalman Smoothing (Maybe?)}
\textcolor{blue}{Figure: plot of score for each method across range of smoothing parameters.  subplots for each ODE \& metric}. 
Add noisy 
\subsection{ODEs: Data Length requirements}
\textcolor{blue}{Figure: plot of score for each method at optimal parameter across range of data length requirements.  subplots for each ODE \& metric}.  \textcolor{red}{Parameter-search wrapper-experiment inside data-length wrapper-experiment inside Data-length comparison comparison experiment}.
\subsection{PDEs}
\textcolor{red}{Use Figure 4 from the E-SINDy paper. Need to implement these}
% \subsection{Model Predictive Control}
% \textcolor{red}{Lorenz stabilization, learn from E-SINDy paper}
\section{Conclusion}
\begin{table}
    \centering
    \begin{tabular}{c|c}
         &  \\
         & 
    \end{tabular}
    \caption{Caption}
    \label{tab:my_label}
\end{table}
\end{document}
